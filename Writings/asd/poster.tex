% Gemini theme
% https://github.com/anishathalye/gemini
%
% We try to keep this Overleaf template in sync with the canonical source on
% GitHub, but it's recommended that you obtain the template directly from
% GitHub to ensure that you are using the latest version.

\documentclass[final]{beamer}

% ====================
% Packages
% ====================

\usepackage{array}
\usepackage{url}
\usepackage{picins}
%\usepackage[utf8]{inputenc}
\usepackage[T1]{fontenc}
\usepackage{lmodern}
\usepackage[size=custom,width=120,height=80,scale=1.0]{beamerposter}
\usetheme{gemini}
\usecolortheme{mit}
\usepackage{graphicx}


% ====================
% Lengths
% ====================

% If you have N columns, choose \sepwidth and \colwidth such that
% (N+1)*\sepwidth + N*\colwidth = \paperwidth
\newlength{\sepwidth}
\newlength{\colwidth}
\setlength{\sepwidth}{0.025\paperwidth}
\setlength{\colwidth}{0.3\paperwidth}

\newcommand{\separatorcolumn}{\begin{column}{\sepwidth}\end{column}}

\pichskip{17pt}% Horizontal gap between picture and text
\renewcommand{\arraystretch}{1.5}
\newcolumntype{P}[1]{>{\centering\arraybackslash}p{#1}}

\title{R and C on Q and F in a Series RLC Circuit}
\author{Edwin P.}
\footercontent{}


\begin{document}

\logoright
{
\includegraphics[scale=0.4]{uni.png}
}

\begin{frame}[t]
\begin{columns}[t]
  \separatorcolumn
  \begin{column}{\colwidth}

    \begin{block}{Introduction}
      Bruh
    \end{block}
    \begin{block}{Methodology}
      Massive Bruh
    \end{block}

  \end{column}

  \separatorcolumn
  \begin{column}{\colwidth}
    \begin{block}{Results}
      Nice
    \end{block}
  \end{column}

  \separatorcolumn
  \begin{column}{\colwidth}

    \begin{block}{Discussion}
      Bruh?
      

    \end{block}

    \begin{block}{Conclusions}
      The equations relating $Q$, $R$ and $C$, and $f_r$ and $C$ for a resistor in series RLC circuit, derived from Ohm's law and reactance equations, were verified up to experimental uncertainty.
    \end{block}

    \begin{block}{Acknowledgements}
      %\nocite{*}
      %\footnotesize{\bibliographystyle{plain}\bibliography{poster}}
      I would like to thank Mr. Kishor for providing circuit components.
    \end{block}
    
    \begin{block}{References}
      asd
    \end{block}
  \end{column}

  \separatorcolumn
\end{columns}
\end{frame}



\end{document}
