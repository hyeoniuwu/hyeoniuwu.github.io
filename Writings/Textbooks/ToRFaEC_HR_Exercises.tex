\documentclass{article}

%FORMATTING
\usepackage{titling}
\usepackage[a4paper, margin=1in]{geometry}
\usepackage[utf8]{inputenc} % "tells the compiler to use the UTF-8 encoding for input files"
% "Without this package, you may encounter errors or unexpected behavior if you use non-ASCII characters in your LaTeX document"
\usepackage{titlesec}  % For adding a dot after section numbers in section headings
\usepackage{mdwlist} % for temporarily escaping out of an enum env

%MATH
\usepackage{amsmath}
\usepackage{amsfonts}
\usepackage{amssymb}
\usepackage{amsthm}
\usepackage{mathtools}
\usepackage{mathrsfs} %For \mathscr
\usepackage{enumerate} % enumerate environment
\usepackage{centernot} % To have \not be centred; EG \centernot\iff

%
\usepackage[colorlinks]{hyperref}
\usepackage{footnotehyper}
\usepackage{parskip} % "inserts a vertical space between paragraphs"

%GRAPHICS
\usepackage{tikz} %for diagrams
\usepackage{fancyhdr} % "customize the headers and footers"
\usepackage{graphicx} % "provides a way to include images"
\usepackage[export]{adjustbox} %For putting frames around images
\usepackage{caption} % use \captionof{} in \center environment

%MISC
\usepackage{pdflscape} % allows creation of landscape pages

\pagestyle{fancy}
\fancyhf{}
\rhead{Edwin P.}
\lhead{\rightmark}
\rfoot{Page \thepage}

\titleformat{\section}{\normalfont\Large\bfseries}{\thesection.}{1em}{} % add a dot after section number
\titleformat{\subsection}{\normalfont\large\bfseries}{\thesubsection.}{1em}{}  % same for the subsection 
\titleformat{\subsubsection}{\normalfont\normalsize\bfseries}{\thesubsubsection.}{1em}{}  % and subsub

% Create title page
\renewcommand\maketitlehooka{\null\mbox{}\vfill}
\renewcommand\maketitlehookd{\vfill\null}

% Footnote options
\renewcommand{\footnoterule}{\noindent\smash{\rule[3pt]{\textwidth}{0.4pt}}}
\renewcommand*{\thefootnote}{(\arabic{footnote})}

\newcommand{\sref}[1]{\textsuperscript{(\ref{#1})}}

%macros
\newcommand{\lv}[1]{\lvert #1\rvert}
\newcommand{\llv}[1]{\left\lvert #1\right\rvert}
\newcommand{\lV}[1]{\lVert #1\rVert}
\newcommand{\llV}[1]{\left\lVert #1\right\rVert}
\newcommand{\lp}[1]{\left(#1\right)}
\newcommand{\lc}[1]{\left\{#1\right\}}
\newcommand{\ls}[1]{\left[#1\right]}
\newcommand{\lan}[1]{\langle #1\rangle}
\newcommand{\llan}[1]{\left\langle #1\right\rangle}
\newcommand{\lcl}[1]{\lceil #1\rceil}
\newcommand{\llcl}[1]{\left\lceil #1\right\rceil}
\newcommand{\lfl}[1]{\lfloor #1\rfloor}
\newcommand{\llfl}[1]{\left\lfloor #1\right\rfloor}

% \newcommand{\seq}[2]{\underset{\left(#2\right)}{\left\langle #1\right\rangle}}
\newcommand{\seq}[2]{\underset{#2}{\left\langle #1\right\rangle}}

\newcommand{\vc}{\vcentcolon}
\newcommand{\lra}{\leftrightarrow}
\newcommand{\Lra}{\Leftrightarrow}

%geometry
\setlength{\parindent}{0pt} % "sets the length of the paragraph indentation to zero"

% Textbook Sections
\theoremstyle{definition}
\newtheorem{thm}{Theorem}[subsubsection]
\newtheorem{defn}{Definition}[subsubsection]
\newtheorem{rmk}{Remark}[subsubsection]
\newtheorem{cor}{Corollary}[subsubsection]
\newtheorem{lem}{Lemma}[subsubsection]
\newtheorem{prop}{Proposition}[subsubsection]
\newtheorem{example}{Example}[subsubsection]
\newtheorem{exercise}{Exercise}[subsubsection]
\newtheorem{prob}{Problem}[section]  % Problem environment
\newenvironment{soln}
  {\begin{proof}[Solution]\setlength{\parskip}{0pt}} %generated by GPT.
  {\end{proof}\vspace{-5pt}} %vspace is to remove space above and below. idk what parskip is doing.

%bibitex options
\nocite{*}
\bibliographystyle{plain}

\allowdisplaybreaks % allows page breaks to occur within certain multiline math environments, such as align, gather, and multline.
%\delimitershortfall=-1pt

\setlength{\jot}{7pt} % sets the vertical spacing between lines in align*, gather, ...

\title{Analysis}
\author{Edwin Park}
\date{2023}

\begin{document}
\clearpage\maketitle\thispagestyle{empty} % Make title
\newpage
\tableofcontents
\newpage\setcounter{page}{1}

\section{Recursive Functions}
The comment on the footnote of $\S1.2$, that intuitionists reject the claim that the function
\begin{align*}
	h(x)=\begin{cases}
		1,&\text{if the Goldbach conjecture is true;}\\
		0,&\text{otherwise;}
	\end{cases}
\end{align*}
is primitive recursive from the reasoning that it must either be true or false, is actually trivial. One just needs to understand that intuitionists have a different notion of ``true'' and ``false''. If one asks me in the midst of a chess game whether I've won, I would answer no, because the game is still going, but that does not mean I have lost; similarly, a statement to an intuitionist is true, false, or contingent/neither, with the latter being due to a non-existence of an explicit construction. The idea behind intuitionism is to define truth/existence/reality to hinge on (somewhat circularly) existence/reality/explicit construction. (Perhaps this is saying even consistent constructions are not ``real'' if not realised in the universe?)
\section{Unsolvable Problems}
\begin{prob}
	Show that the function
	\begin{align*}
		g(x)=\begin{cases}
			1,&\text{if a consecutive run of at least $x$ 5's occurs in the decimal expansion of $\pi$;}\\
			0,&\text{otherwise;}
		\end{cases}
	\end{align*}
	is primitive recursive.
\end{prob}
\begin{soln}
	We just need to find an algorithm for $\pi$ that is primitive recursive; the BBP-formula trivially solves this. (Actually, I think the question wants you to explicitly develop a primitve parsing algorithm, supposing that $\pi$ is already given. But whatever.)
\end{soln}
\begin{prob}
	Define $f$ by
	\begin{align*}
		f(x)=\begin{cases}
			1,&\text{if $\varphi_x(x)=1$ $\pi$;}\\
			0,&\text{otherwise.}
		\end{cases}
	\end{align*}
	Is $f$ recursive?
\end{prob}
\begin{soln}
	Immediately this reeks of being unrecursive. Suppose that $f$ is recursive. Then $g=\neg f$ is also recursive. Let $g=\varphi_y$. Is $g(y)=0$? Then $\varphi_y(y)=0$ and $\varphi_y(y)=1$; if $g(y)=1$ we get a similar contradiction.
\end{soln}
\begin{prob}
	Consider the list of primitive recursive derivations decribed in $\S 1.4$. Let $f_x$ be the primitive recursive function determined by the $(x+1)$st derivation in this list $x=0,1,2,\dots$. Define $g=\lambda xy[f_x(y)]$. Is $g$ recursive? Is $g$ primitive recursive?
\end{prob}
\begin{soln}
	$g$ is primitive recursive, as it is effectively the concatenation of two primitive recursive operations; finding $f_x$ and computing $f_x$. $f_x$ may be found easily by iterating through the method by which the enumeration was generated (e.g iterating through symbols).
\end{soln}

\end{document}