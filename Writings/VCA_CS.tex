\documentclass{article}
\usepackage[utf8]{inputenc}
\usepackage[a4paper, margin=1in]{geometry}
\usepackage{amsmath}
\usepackage{amsfonts}
\usepackage{amssymb}
\usepackage{amsthm}
\setlength{\parindent}{0pt}
\usepackage{titling}
\usepackage{graphicx}
\usepackage{pgf}
\usepackage{fancyhdr}
\usepackage{mathtools}
\usepackage{pdflscape}
\usepackage[colorlinks]{hyperref}
\usepackage{footnotehyper}
\usepackage{parskip}
\usepackage{centernot}

\pagestyle{fancy}
\fancyhf{}
\rhead{Edwin P.}
\lhead{\rightmark}
\rfoot{Page \thepage}

\renewcommand\maketitlehooka{\null\mbox{}\vfill}
\renewcommand\maketitlehookd{\vfill\null}

\renewcommand{\footnoterule}{\noindent\smash{\rule[3pt]{\textwidth}{0.4pt}}}
\renewcommand*{\thefootnote}{(\arabic{footnote})}

\newcommand{\sref}[1]{\textsuperscript{(\ref{#1})}}

\theoremstyle{definition}
\newtheorem{thm}{Theorem}[subsubsection]
\newtheorem{defn}{Definition}[subsubsection]
\newtheorem{rmk}{Remark}[subsubsection]
\newtheorem{cor}{Corollary}[subsubsection]
\newtheorem{lem}{Lemma}[subsubsection]
\newtheorem{prop}{Proposition}[subsubsection]

%bibitex options
\nocite{*}
\bibliographystyle{plain}

\allowdisplaybreaks
%\delimitershortfall=-1pt

\setlength{\jot}{7pt}

\title{Analysis}
\author{Edwin Park}
\date{2023}

\begin{document}

\clearpage\maketitle\thispagestyle{empty}
\newpage
\tableofcontents
\newpage\setcounter{page}{1}
\section{Taylor Polynomials}
\begin{align*}
	f(\vec x)&=f(\vec{x_0}+(\vec x-\vec{x_0}))\\
	&\approx f(\vec{x_0})+D_f(\vec{x_0})(\vec x-\vec{x_0})+\frac{1}{2}(\vec x-\vec{x_0})^\top H_f(\vec{x_0})(\vec x-\vec{x_0})\\
	&=f(\vec{x_0})+\begin{bmatrix}f_x(\vec{x_0})&f_y(\vec{x_0})\end{bmatrix}\begin{bmatrix}x-x_0\\y-y_0\end{bmatrix}+\frac{1}{2}\begin{bmatrix}x-x_0&y-y_0\end{bmatrix}\begin{bmatrix}f_{xx}(\vec{x_0})&f_{xy}(\vec{x_0})\\f_{yx}(\vec{x_0})&f_{yy}(\vec{x_0})\end{bmatrix}\begin{bmatrix}x-x_0\\y-y_0\end{bmatrix}
\end{align*}
Say we wanted to find an upper bound for the error of a first-order approximation at $\vec x$. The error then is equal to the quadratic term evaluated at some point in between the central point, $\vec{x_0}$, and the point of interest, $\vec x$. In particular, it is, for some $t\in[0,1]$,
\begin{align*}
	\frac{1}{2}\left[D_{\vec x-\vec{x_0}}^2f\right](t\vec x+(1-t)\vec{x_0})&=\frac{1}{2}\begin{bmatrix}x-x_0&y-y_0\end{bmatrix}\begin{bmatrix}f_{xx}(t\vec x+(1-t)\vec{x_0})&f_{xy}(t\vec x+(1-t)\vec{x_0})\\f_{yx}(t\vec x+(1-t)\vec{x_0})&f_{yy}(t\vec x+(1-t)\vec{x_0})\end{bmatrix}\begin{bmatrix}x-x_0\\y-y_0\end{bmatrix}.
\end{align*}
\section{Integral Theorems}
We may specify a surface by
\begin{align*}
	(x,y,z)=\Phi(a,b).
\end{align*}
\begin{thm}
	\begin{align*}
		\iint_{\Phi(D)}f\,dS=\iint_Df(\Phi(a,b))\,\lVert\Phi_a\times\Phi_b\rVert\,da\,db
	\end{align*}
\end{thm}
More generally one has the multivariable change of variables formula:
\begin{align*}
	\int_{\Phi(D)}f(\vec r)\,d\vec r=\int_Df(\Phi(\vec r))\det D\Phi\,d\vec r.
\end{align*}
\begin{thm}[Green's Theorem]
	\begin{align*}
		\int_D\partial_xY-\partial_yX\,dx\,dy=\int_{\partial D}\begin{bmatrix}X\\Y\end{bmatrix}\cdot d\vec r
	\end{align*}
\end{thm}

\begin{thm}[Divergence Theorem (2D)]
	\begin{align*}
		\int_C\vec F\cdot\hat{n}\,ds=\iint_D\nabla\cdot\vec F\,dx\,dy;
	\end{align*}
	This is equivalent to Green's theorem.
\end{thm}
\begin{thm}[Stokes']
	\begin{align*}
		\iint_\Sigma(\nabla\times\vec F)\cdot d\vec S=\int_{\partial\Sigma}\vec F\cdot ds
	\end{align*}
\end{thm}
\begin{thm}[Gauss' Divergence Theorem]
	\begin{align*}
		\iiint_\Omega\nabla\cdot\vec F\,dV=\iint_{\partial\Omega}\vec F\cdot\,d\vec S
	\end{align*}
\end{thm}

\section{Spherical}
\begin{center}
	\includegraphics[width=0.5\textwidth]{spherical.png}
\end{center}
\begin{align*}
	x&=r\sin(\phi)\cos(\theta)\\
	y&=r\sin(\phi)\sin(\theta)\\
	z&=r\cos(\phi)\\
	\det D\Phi&=r^2\sin(\phi)\\
	\theta&\in[0,2\pi)\\
	\phi&\in[0,\pi)
\end{align*}
\end{document}