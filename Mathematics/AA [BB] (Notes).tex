\documentclass{article}
\usepackage[utf8]{inputenc}
\usepackage[a4paper, margin=1.2in]{geometry}
\usepackage{amsmath}
\usepackage{amsfonts}
\setlength{\parindent}{0pt}
\usepackage{titling}
\usepackage{graphicx}
\usepackage{pgf}
\usepackage{fancyhdr}
\usepackage{mathtools}

\pagestyle{fancy}
\fancyhf{}
\rhead{Edwin P.}
\lhead{\rightmark}
\rfoot{Page \thepage}

\renewcommand\maketitlehooka{\null\mbox{}\vfill}
\renewcommand\maketitlehookd{\vfill\null}
\renewcommand\({\left(}
\renewcommand\){\right)}

\allowdisplaybreaks
%\delimitershortfall=-0.5pt

\setlength{\jot}{7pt}

\title{Project}
\author{Edwin Park}
\date{July 2022}

\begin{document}

\clearpage\maketitle\thispagestyle{empty}
\newpage
\tableofcontents
\newpage\setcounter{page}{1}

\section{Integers}
\subsection{Divisors}
\subsection{Primes}
\subsection{Congruences}
\underline{\textbf{Theorem.}} (Chinese Remainder Theorem)
\\
\vspace{2mm}
\\
Given:
\begin{align*}
    x\equiv a\pmod{n}\\
    x\equiv b\pmod{m}\\
    \textnormal{gcd}(m,n)=1,
\end{align*}
a solution is given by
\begin{align*}
    x=arm+bsn,
\end{align*}
where $rm+sn=1$.

\vspace{24mm}
\subsection{Integers Modulo N}
\underline{\textbf{Definition.}} Euler's totient functions is defined as:
\begin{align*}
    \varphi(n)=\#k\,|\,(k<n\land (n,k)=1).
\end{align*}
For any $n\in\mathbb{N}$:
\begin{align*}
    \varphi(n)=\prod_{i}{\left(1-\frac{1}{p_i}\right)}
\end{align*}
\underline{\textbf{Theorem.}} (Euler)
\begin{align*}
    (a,n)=1\Rightarrow a^{\varphi(n)}\equiv 1\pmod{n}
\end{align*}
\newpage

\section{Functions}
\subsection{Functions}
\subsection{Equivalence Relations}
\underline{\textbf{Notation.}} The collection of equivalence classes of S under relation $\sim$ is denoted by $S/\sim$.

\vspace{24mm}
\subsection{Permutations}
\underline{\textbf{Definition.}} A permutation of a set S is a bijection $\sigma: S\rightarrow S$. A cycle of length $k$ is a permutation satisfying:
\begin{align*}
    \sigma(a_i)&=a_{(i+1)\%k}\\
    \sigma(x)&=x, \textnormal{   otherwise}
\end{align*}
where $a_1, \dots, a_k\in S$. Two cycles are disjoint if their respective $\{a_i\}$ have no overlaps.

\vspace{6mm}
\underline{\textbf{Theorem.}} Every cycle in $S_n$ may be written as a unique product of disjoint cycles.

\vspace{6mm}
\underline{\textbf{Theorem.}} Transposition-decompositions of permutations always have the same parity.
\newpage

\section{Groups}
\subsection{Definition of a Group}
\underline{\textbf{Definition.}} A set $G$ is a group under an operation $*$ if $*$ is an associative binary operation on $G$, $G$ contains an identity element, and $G$ is closed under inversion.
\subsection{Subgroups}
\underline{\textbf{Theorem.}} (Lagrange) 
\begin{align*}
    H\leq G\Rightarrow|H|\mid|G|
\end{align*}
\subsection{Constructing Examples}
\underline{\textbf{Property.}} 
\begin{align*}
    H, K\leq G\land h^{-1}kh\in K\Rightarrow HK\leq G
\end{align*}
\subsection{Isomorphisms}
\subsection{Cyclic Groups}
\underline{\textbf{Theorem.}} Every subgroup of a cyclic group is cyclic.
\subsection{Permutation Groups}
\underline{\textbf{Theorem.}} (Cayley) Every group is isomorphic to a permutation group (subgroup of $S_n$).
\subsection{Homomorphisms}
\underline{\textbf{Definition.}} A subgroup H of G is normal iff $ghg^{-1}\in H$.
\subsection{Cosets, Normal Subgroups, and Factor Groups}
\underline{\textbf{Definition.}} A coset takes the form $aH$ where $H\leq G$. The index is the number of cosets; \[[G:H]\vcentcolon=\frac{|G|}{|H|}.\] 

\vspace{6mm}
\underline{\textbf{Theorem.}} The set of cosets of a normal subgroup form a group under multiplication. The group of cosets of a normal subgroup is called the factor group of $G$ determined by $N$, denoted by $G/N$. Note that left and right cosets of a normal subgroup are always equivalent.

\vspace{6mm}
\underline{\textbf{Theorem.}} Where $N$ is a normal subgroup of G, there is a bijection between subgroups of $G/N$ and subgroups of $G$ containting $N$. 

\vspace{3mm}
\emph{Proof.} First we note that homomorphisms preserve subgroups, and normality if surjective. Furthermore, the pseudo-inverse operation of the homomorphism $\phi: G_1\rightarrow G_2$, \[\phi^{-1}(H_2\leq G_2)=\{x\in G_1\mid \phi(x)\in H_2\},\] preserves subgroups and normality.\\
The bijection is given by the natural projection $\pi: G\rightarrow G/N,\quad \pi(x)=xN$. Note that the actual bijection, using a slight abuse of notation, is: 
\begin{align*}
    \pi(H\leq G)=\bigcup_{x\in H} \{xN\}.
\end{align*}
Note that in the pseudo-inverse function, $\pi^{-1}(K\leq G/N)=\{x\in G\mid xN\in K\}$, the output always contains $N$ as it is the identity in $G/N$ (and in turn $K$ contains it). To show bijectivity, ... cbf

\vspace{6mm}
\underline{\textbf{Theorem.}} (Fundamental Homomorphism Theorem) 
\begin{align*}
    G/\text{ker}(\phi)\cong\phi(G),
\end{align*}
where $\phi:G\rightarrow H$ is a homomorphism, $G$ and $H$ are groups. 
\newpage

\section{Polynomials}
\subsection{Fields; Roots of Polynomials}
\underline{\textbf{Definition.}} A field is a set F which is a group under two binary operations $+$ and $*$, which also satisfy distributivity.

\vspace{6mm}
Multiplication of polynomial coefficients:
\begin{align*}
    \left(\sum_{i=0}^{m}a_ix^i\right)\left(\sum_{i=0}^{n}b_ix^i\right)=\sum_{i=0}^{m+n}\left(\sum_{j=0}^{i}a_jb_{i-j}\right)x^i
\end{align*}
\subsection{Factors}
\subsection{Existence of Roots}
$\langle p(x)\rangle$ denotes the set of all polynomials divisible by $p(x)$. 

\vspace{6mm}
\underline{\textbf{Theorem.}} Given that $p(x)$ is non-constant, $F[x]/\langle p(x)\rangle$ is a field iff. $p(x)$ is irreducible over $F$.

\vspace{6mm}
The congruence class $[x]$ in $F[x]/\langle p(x)\rangle$ satisfies $p([x])=[0]$.
\subsection{Polynomials over Z, Q, R and C}

\newpage
\section{Commutative Rings}
\subsection{Commutative Rings; Integral Domains}
\underline{\textbf{Definition.}} A commutative ring is a field without the requirement that inversion is closed with respect to $*$.

\vspace{6mm}
A subring must share identities with its parent.

\vspace{6mm}
\underline{\textbf{Definition.}} An integral domain is a commutative ring where $1\not=0$ and the product of non-zero elements is always non-zero.

\vspace{6mm}
\underline{\textbf{Theorem.}} Any subring of a field is an integral domain.

\vspace{6mm}
\underline{\textbf{Theorem.}} Any finite integral domain is a field.
\subsection{Ring Homomorphisms}
\underline{\textbf{Definition.}} The characteristic of a commutative ring is the minimal natural number satisfying $n*1=0$; if no such number exists, then the characteristic is zero.

\vspace{6mm}
\underline{\textbf{Definition.}} A ring homomorphism preserves sums, products and the identity.
\subsection{Ideals and Factor Rings}
\underline{\textbf{Definition.}} An ideal $I$ is a non-empty subset of a commutative ring $R$ closed under addition, subtraction and multiplication by any element in $R$.

\vspace{6mm}
\underline{\textbf{Theorem.}} $R/I$ is a commutative ring, called the factor ring of $R$ modulo $I$.

\vspace{6mm}
\underline{\textbf{Theorem.}} Given that $I$ is a proper ideal of the commutative ring $R$,
\begin{align*}
    R/I\text{ is a field}&\iff I\text{ is maximal}\\
    R/I\text{ is an integral domain}&\iff I\text{ is prime}.\\
\end{align*}
\subsection{Quotient Fields}


\end{document}